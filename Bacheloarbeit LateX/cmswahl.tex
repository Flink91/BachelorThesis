\chapter{Die Wahl des richtigen Content Management Systems}
\label{kap-cmswahl}


Die Wahl des richtigen CMS kann aufgrund der F�lle von m�glichen Optionen sehr schwierig sein. Man m�chte das beste Tool f�r den Job finden. Hier kann die Webseite CMSMatrix eine Hilfe sein. Sie listet �ber 1000 verschiedene Systeme, die sich einzeln ausw�hlen und anhand s�mtlicher wichtiger Kriterien vergleichen lassen. Dar�ber hinaus ist es sicherlich eine gute Idee sich an die popul�reren Optionen zu halten, da man hier mit dem besten Support und der besten Community rechnen kann. Selbst unter den f�nf popul�rsten CMS gibt es immense Unterschiede, die zusammen den Gro�teil der Bed�rfnisse abdecken.\\ 


\section{Die Vorauswahl dreier geeigneter Content Management Systeme}
http://www.cmsmatrix.org/\\

Die Vorauswahl von geeigneten Content Management Systemen f�r das WegE Projekt wurde durch mehrere Faktoren beeinflusst. Die gr��te Rolle spielt dabei zweifelsohne das Rechenzentrum der Universit�t Bamberg, die dort schon bestehenden Systeme und die vorhandene Expertise zur langfristigen Betreuung des WegE Projekts. Vor Allem mit Typo3 hat das Rechenzentrum Erfahrung, Sharepoint wird seit kurzem jedoch auch genutzt. Die sehr popul�ren Optionen Joomla! und Drupal w�ren durchaus geeignet f�r das Projekt und werden rein durch den Aspekt der Vorkenntnis ausgeschlossen. Die Gr��e und Ambiguit�t des Projekts schlie�t au�erdem das popul�rste aller CMS Wordpress aus, welches sich eher f�r kleinere Projekte eignet. Somit landete die Vorauswahl bei Sahrepoint, Typo3 und LiveRay, welche nun vorgestellt und gepr�ft werden.

\section{Sharepoint}

Sharepoint ist ein Produkt von Microsoft und wurde haupts�chlich f�r die Verwendung innerhalb von Unternehmen entwickelt. Da es nur im Browser l�uft gibt es hier keine Einschr�nkung auf Windows, wie man vielleicht denken k�nnte, auch wenn das Standarddesign hierauf optimiert wurde. Hier dient es als gemeinsamer Ort zum Speichern, Strukturieren und Freigeben von Dateien und Informationen. (https://support.office.com/de-de/article/Was-ist-SharePoint-97b915e6-651b-43b2-827d-fb25777f446f) Daf�r verwendet Sharepoint einige Sharepoint einige Sharepoint-Technologien:
\begin{itemize}

\item SharePoint Online l�sst ein Unternehmen einen Cloud-Dienst von Microsoft verwenden, anstatt einen eigenen Server aufzusetzen. 
\item SharePoint Foundation ist die grundlegende Technologie f�r die SharePoint Webseiten, welche in verschiedenster Form auftreten k�nnen. Am prominentesten nat�rlich Seiten zur Zusammenarbeit an Listen, Daten und Dokumenten, aber au�erdem noch Blogs und Wikis. Foundation ist eine kostenlose Edition von SharePoint.
\item SharePoint Server ist die kostenpflichtige Edition und bietet selbstverst�ndlich weit mehr Features als SharePoint Foundation. Diese sind zum Beispiel Enterprise Content Management, Business Intelligence, unternehmensweite Suche, pers�nliche Websites und ein Newsfeed. Die Kosten hierf�r lassen sich schwer absch�tzen, da dabei eine Reihe von Variablen eine Rolle spielen. Die reinen Kosten f�r die Lizensierung belaufen sich zum Stand der Arbeit auf 4,20 Euro bis 8,40 Euro pro Benutzer pro Monat.
\item SharePoint Designer erlaubt es leicht Layouts, Workflows und Verbindungen zu externen Datenquellen anzupassen.
\end{itemize}

\section{LiveRay Portal}

https://www.liferay.com/products/liferay-portal/nav/overview

Das Hauptprodukt von Liveray hei�t Liveray Portal und hat den Fokus mehr auf Businessl�sungen und Gesch�ftsprozessen. Es ist lizenzkostenfrei und Open Source.  Die Kernfunktionalit�ten sind die Authentifizierung von Nutzern, ....

\section{Typo3}

Wer die Gliederung gelesen hat, wird sicherlich bemerkt haben, dass die Wahl auf Typo3 fiel und diesem CMS die n�chsten zwei Kapitel gewidmet sind, weshalb alles Weitere dort erkl�rt wird.


