\chapter{Die Wahl des richtigen Content Management Systems}
\label{kap-cmswahl}

Zum vollen Verst�ndnis der Arbeit werden grundlegende Kenntnisse des Software Engineering und der Web Technologien vorausgesetzt. Weitere essentielle Begrifflichkeiten und Technologien werden nun erkl�rt.

\section{Befriffskl�rung CMS/WCMS}

Ein Content Management System (Abk. CMS), oder auf Deutsch Inhaltsverwaltungssystem, ist eine Software, die bei der Erstellung, Pflege und Planung von Content helfen kann. Vor Allem dann, wenn mehrere Leute an einem Projekt zusammenarbeiten. Heutzutage trifft man solche CMS zumeist im Web, woraus sich der Begriff Web Content Management System, WCMS, ergibt. Diese erm�glichen konkret die Erstellung und Bearbeitung multimedialer Inhalte auf Webseiten ohne Programmierkenntnisse. So kann beispielsweise ein Journalist ohne viel M�he News auf einer Webseite ver�ffentlichen. 
Content Management Systeme umfassen meist folgende Features:

\begin{itemize}
	\item M�glichkeit, unterschiedliche Rollen und Verantwortlichkeiten an verschiedene Nutzer und Content-Kategorien/Typen zu vergeben
	\item Identifizieren der m�glichen Nutzer und ihrer Rollen
	\item Definition der Verarbeitungsprozesse als Workflow
	\item Erstellung und Verwaltung von Templates
	\item Semantisches Ordnen von Inhalten
	\item Ver�ffentlichung von Content
\end{itemize}

Obwohl WCMS der pr�zisere Begriff ist, werden diese aufgrund ihrer Verbreitung oft mit dem Oberbegriff CMS betitelt und auch in dieser Arbeit synonym verwendet. 
Fast immer gliedern sich Content Management Systeme in ein Backend und ein Frontend. Im Backend k�nnen sich nur bestimmte Nutzer, wie Administratoren und Autoren einloggen um hier die Seite und deren Inhalte zu verwalten. Daf�r ist kein extra Programm n�tig, das Backend l�sst sich bequem durch den Browser erreichen. Das Frontend ist die Webseite die �ffentlich zug�nglich ist und jene Inhalte f�r die Besucher der Webseite pr�sentiert.
Die Liste an bestehender CMS Software ist sehr lang. Die meistverwendeten CMS sind momentan Wordpress, Drupal  und Joomla.


\section{Die Vorauswahl dreier geeigneter Content Management Systeme}
http://www.cmsmatrix.org/
Die Vorauswahl von geeigneten Content Management Systemen wurde durch mehrere Faktoren beeinflusst. Die gr��te Rolle spielt dabei zweifelsohne das Rechenzentrum der Universit�t Bamberg, die dort schon bestehenden Systeme und die vorhandene Expertise zur langfristigen Betreuung des WegE Projekts. 

\section{Sharepoint}

Sharepoint ist ein Produkt von Microsoft und wurde haupts�chlich f�r die Verwendung innerhalb von Unternehmen entwickelt. Da es nur im Browser l�uft gibt es hier keine Einschr�nkung auf Windows, wie man vielleicht denken k�nnte, auch wenn das Standarddesign hierauf optimiert wurde. Hier dient es als gemeinsamer Ort zum Speichern, Strukturieren und Freigeben von Dateien und Informationen. (https://support.office.com/de-de/article/Was-ist-SharePoint-97b915e6-651b-43b2-827d-fb25777f446f) Daf�r verwendet Sharepoint einige Sharepoint einige Sharepoint-Technologien:
\begin{itemize}

\item SharePoint Online l�sst ein Unternehmen einen Cloud-Dienst von Microsoft verwenden, anstatt einen eigenen Server aufzusetzen. 
\item SharePoint Foundation ist die grundlegende Technologie f�r die SharePoint Webseiten, welche in verschiedenster Form auftreten k�nnen. Am prominentesten nat�rlich Seiten zur Zusammenarbeit an Listen, Daten und Dokumenten, aber au�erdem noch Blogs und Wikis. Foundation ist eine kostenlose Edition von SharePoint.
\item SharePoint Server ist die kostenpflichtige Edition und bietet selbstverst�ndlich weit mehr Features als SharePoint Foundation. Diese sind zum Beispiel Enterprise Content Management, Business Intelligence, unternehmensweite Suche, pers�nliche Websites und ein Newsfeed. Die Kosten hierf�r lassen sich schwer absch�tzen, da dabei eine Reihe von Variablen eine Rolle spielen. Die reinen Kosten f�r die Lizensierung belaufen sich zum Stand der Arbeit auf 4,20? bis 8,40? pro Benutzer pro Monat.
\item SharePoint Designer erlaubt es leicht Layouts, Workflows und Verbindungen zu externen Datenquellen anzupassen.
\end{itemize}

\section{LiveRay}

\section{Typo3}


