\chapter{Bootstrap Fluidtemplate}


\paragraph{index.html}

In der Index.html wird das Men� folgenderma�en eingebunden.
\begin{verbatim}
	[...]
	<div id="navbar" class="collapse navbar-collapse">
	  <f:cObject typoscriptObjectPath="lib.mainmenu" />
	</div>
	[...]
\end{verbatim}

Der restliche Content wird f�r die Zwecke dieser Arbeit nicht speziell behandelt. Folgender Code in der index.html reicht aus.
\begin{verbatim}
<f:format.raw>{content}</f:format.raw>
\end{verbatim}


\paragraph{setup.ts}

Die setup.ts f�r das simple Bootstrap Men�.
\begin{verbatim}
	config.contentObjectExceptionHandler = 0
	# Main Menu
	lib.mainmenu = HMENU
	lib.mainmenu{
	 entryLevel = 0
	 1 = TMENU
	 1{
 	  wrap = <ul class="nav navbar-nav"> | </ul>
	  noBlur = 1
	  NO = 1
	  NO{
	   wrapItemAndSub = <li> | </li>
	   stdWrap.htmlSpecialChars = 1
	   ATagTitle.field = title
	  }
	  ACT <.NO
	  ACT{
	   wrapItemAndSub = <li class="active"> | </li>
	  }
	 }
	}
	#Submenu, 2nd layer
	lib.submenu = HMENU
	lib.submenu{
	 entryLevel = 1
	 1 = TMENU
	 1{
	  wrap = <ul class="nav navbar-nav"> | </ul>
	  noBlur = 1
	  NO = 1
	  NO{
	   wrapItemAndSub = <li> | </li>
	   stdWrap.htmlSpecialChars = 1
	   ATagTitle.field = title
	  }
	  ACT <.NO
	  ACT{
	   wrapItemAndSub = <li class="active"> | </li>
	  }
	 }
	}
	
	page = PAGE
	page{
	 includeCSS.bootstrap = fileadmin/Template_WegE/bootstrap/css/bootstrap.min.css
	 includeCSS.style = fileadmin/Template_WegE/bootstrap/css/style.css
	 includeJSFooter.jquery = fileadmin/Template_WegE/bootstrap/js/jquery-2.2.3.min.js
	 includeJSFooter.bootstrapjs = fileadmin/Template_WegE/bootstrap/js/bootstrap.min.js
	
	 10 = FLUIDTEMPLATE
	 10{
	  file = fileadmin/Template_WegE/index.html
	  layoutRootPath = fileadmin/Template_WegE/layouts/
	  partialRootPath = fileadmin/Template_WegE/partials/
	
	  variables{
	   siteName = TEXT
	   siteName.value = WegE Fallstudie
	
	   content < styles.content.get
	  }
	 }
	}
\end{verbatim}