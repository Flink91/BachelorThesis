\chapter{Validierung der Requirements basierend auf der Implementation}
Nachdem die Umsetzung der Liste von Requirements aus Kapitel \ref{kap-projektanalyse} beendet ist, k�nnen diese Requirements validiert werden. Die Validierung geschieht in diesem Kapitel anhand der Kriterien Korrektheit,  Machbarkeit und Notwendigkeit.\\
Machbarkeit untersucht mit Hilfe des Wissens �ber die Technologie, ob die gew�hlten Requirements tats�chlich umgesetzt werden k�nnen. Dabei sollten Budget und Zeit mit in Betracht gezogen werden.
Notwendigkeit untersucht ob einige der Requirements eventuell nicht umgesetzt werden m�ssen. \cite{Sommerville2007}

\begin{itemize}
	\item \textbf{Statische Informationen}\\
	Dieser Punkt klingt trivial, doch manche Systeme erf�llen dieses Requirement besser als andere. In Typo3 lassen sich statische Webseiten leicht erstellen und kompromisslos anzeigen. Lediglich das Backend k�nnte f�r Redakteure zu Beginn etwas einsch�chternd sein, weshalb eventuell etwas anf�ngliche Unterst�tzung aus der IT Abteilung n�tig ist. Die Machbarkeit ist dennoch gegeben. Die Notwendigkeit daf�r ist auch sehr hoch, da die �ffentlichkeit in das WegE Projekt mit einbezogen werden soll. 
	
	\item \textbf{Newsfeed}\\
	Die Implementation eines Newsfeed stellte sich als recht einfach heraus. Die news Extension l�uft stabil und l�sst sich nach Bedarf sogar mit vielen weiteren Extensions erweitern/verbinden. F�r einen einfachen Newsfeed ist die Machbarkeit absolut gegeben. Ein Newsfeed hat dieselbe Aufgabe, wie statische Informationen, doch bietet stetige aktuelle Informationen strukturierter an, als solche. Wieder geht es darum Informationen nach au�en zu tragen, was die Notwendigkeit hoch einstuft.
	
	\item \textbf{Blogs}\\
	
	\item \textbf{Redakteure}\\
	
	
	
	\item \textbf{Forum}\\
	
	\item \textbf{Registrierung}\\
	
	\item \textbf{Forschungsergebnisse teilen}\\
	
	\item \textbf{Mail Formular}\\
	
	\item \textbf{Kalender}\\
	
	\item \textbf{Search Engine Optimization}\\
	
\end{itemize}

