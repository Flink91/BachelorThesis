\chapter{Einleitung}

Moderne Webanwendungen und Plattformen beinhalten heutzutage weitaus mehr als die M�glichkeit statische Informationen anzuzeigen. Mit dem Aufkommen des Web 2.0 hat das Internet einen gewaltigen Sprung gemacht und ein weiteres Mal, ohne �bertreibung, die Welt ver�ndert. Und wenn es auch f�r einige Leute immer noch Neuland ist, sind die Vorz�ge des Internets inzwischen auch bei den traditionelleren Institutionen mit hohem Ansehen vertreten. Die Rede ist von staatlichen Institutionen. So kommt es, dass der Staat Deutschland 2013 ein massives Projekt zur Lehrerbildung ins Leben gerufen hat, das im Namen des WegE Projekts eine staatlich gef�rderte Online-Pr�senz dazugewinnt.

\section{Das WegE Projekt}

Das WegE Projekt der Universit�t Bamberg\footnote{http://www.uni-bamberg.de/wege} ist ein Teil der 'Qualit�tsoffensive Lehrerbildung'. WegE steht f�r 'Wegweisende Lehrerbildung' und hat als Hauptziel die Entwicklung reflexiver Kommunikationsprozesse. Die fachliche Zusammenarbeit zwischen Wissenschaftlern und Lehrern, sowie Lehramtsstudenten soll verbessert werden. Die Lehrerbildung an den fachlichen St�rken der Universit�t Bamberg wird profiliert, die Zusammenarbeit von Fachwissenschaften und Schulen wird verbessert, die Fortbildungsangebote werden verst�rkt und das Gesamtprojekt wird f�r zuk�nftige Vorhaben aufw�ndig evaluiert. All dies f�hrt zu dem simpleren, indirekten Ziel der Verbesserung der Schulbildung. Dabei besteht das WegE Projekt im Wesentlichen aus vier einzelnen Projektvorhaben:
\begin{description}
	\item{\emph{KulturPLUS}}
	
	Die Organisation KulturPLUS wird sich der Vernetzung geistes- und kulturwissenschaftlicher Perspektiven widmen. Dieses Vorhaben geht von den Fakult�ten Geistes- und Kulturwissenschaften und Humanwissenschaften, genauer der Evangelischen Theologie, aus. Durch verschiedene M�glichkeiten soll Lehramtsstudenten die Kompetenz zum Umgang mit den Herausforderungen der kulturellen Vielfalt bez�glich sprachlicher, historischer, geographischer und relig�ser Kontexte, beigebracht werden. Die konkrete Umsetzung dieser Ziele ist durch verschiedene Optionen vorgesehen. Darunter ein Wahlpflichtkurs namens ?KulturPLUS-Modul?, die �berarbeitung von Schulpraktika, die Ver�ffentlichung fachwissenschaftlicher Ergebnisse auf der WegE Plattform und viele mehr.

\item{\emph{BilVer}}

	BilVer (\textsc{Bil}dungswissenschaft im \textsc{Ver}bund)\cite{BilVer} k�mmert sich um die fallbezogene Vernetzung der bildungswissenschaftlichen Ausbildungsstelle.

\item{\emph{BERA}}

hat als Ziel den Aufbau eines Kompetenzzentrums in Bamberg. Dessen Aufgabe wird die St�rkung der beratungsbezogenen Professionsanteile im Studium und die Kooperation mit den Schulen der Region sein.
\item{\emph{BeBi}}

BeBi, kurz f�r Berufliche Bildung wird sich mit der Profilierung einiger Studieng�nge besch�ftigen.
\end{description}


In die Qualit�tsoffensive der Lehrerbildung investiert der Bund bis 2023 insgesamt eine halbe Milliarde Euro bis zum Jahre 2023. Der Zeitraum f�r die F�rderung des WegE Projekts ist weniger lang. Diese begann am 01.01.2016 und wird bis zum 30.09.2016 bestehen bleiben.



\section{Zielsetzung und Vorgehensweise}

Das grundlegende Ziel dieser Arbeit wird aus dem konkreten Projekt der wegweisenden Lehrerbildung motiviert, wird jedoch auch einen allgemeineren Ton anschlagen, um so eine Relevanz f�r technisch �hnliche Projekte zu erreichen. In den folgenden Kapiteln werden zun�chst die Anforderungen an das System behandelt. Anschlie�end werden verschiedene Content Management Systeme hinsichtlich dieser Anforderungen verglichen und gepr�ft. Im Hauptteil wird die technische Umsetzung der einzelnen Anforderungen genau gepr�ft und getestet. Hierzu wird vor Allem die Eignung bereits bestehender Extensions/Plugins untersucht. 


\section{Aufbau der Arbeit}

In Kapitel~\ref{kap-problemstellung} wird das Problem der Weltherrschaftserringung
genauer erl�utert, das eine Grundlage f�r die Evaluation von
Weltherrschafts�bernahmestrategien darstellt. Dabei werden die Aufgabenstellungen und die
damit verbundenen Probleme und Herausforderungen beschrieben.

...
