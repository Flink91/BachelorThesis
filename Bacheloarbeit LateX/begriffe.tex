\chapter{Begriffskl�rungen}

Zum vollen Verst�ndnis der Arbeit werden grundlegende Kenntnisse des Software Engineering und der Web Technologien vorausgesetzt. Weitere essentielle Begrifflichkeiten und Technologien werden nun erkl�rt.

\section{CMS und WCMS}

Ein Content Management System (Abk. CMS), oder auf Deutsch Inhaltsverwaltungssystem, ist eine Software, die bei der Erstellung, Pflege und Planung von Content helfen kann. Vor Allem dann, wenn mehrere Leute an einem Projekt zusammenarbeiten. Heutzutage trifft man solche CMS zumeist im Web, woraus sich der Begriff Web Content Management System, WCMS, ergibt. Diese erm�glichen konkret die Erstellung und Bearbeitung multimedialer Inhalte auf Webseiten ohne Programmierkenntnisse. So kann beispielsweise ein Journalist ohne viel M�he News auf einer Webseite ver�ffentlichen. 
Content Management Systeme umfassen meist folgende Features:

\begin{itemize}
	\item M�glichkeit, unterschiedliche Rollen und Verantwortlichkeiten an verschiedene Nutzer und Content-Kategorien/Typen zu vergeben
	\item Identifizieren der m�glichen Nutzer und ihrer Rollen
	\item Definition der Verarbeitungsprozesse als Workflow
	\item Erstellung und Verwaltung von Templates
	\item Semantisches Ordnen von Inhalten
	\item Ver�ffentlichung von Content
\end{itemize}

Obwohl WCMS der pr�zisere Begriff ist, werden diese aufgrund ihrer Verbreitung oft mit dem Oberbegriff CMS betitelt und auch in dieser Arbeit synonym verwendet. 
Fast immer gliedern sich Content Management Systeme in ein Backend und ein Frontend. Im Backend k�nnen sich nur bestimmte Nutzer, wie Administratoren und Autoren einloggen um hier die Seite und deren Inhalte zu verwalten. Daf�r ist kein extra Programm n�tig, das Backend l�sst sich bequem durch den Browser erreichen. Das Frontend ist die Webseite die �ffentlich zug�nglich ist und jene Inhalte f�r die Besucher der Webseite pr�sentiert.
Die Liste an bestehender CMS Software ist sehr lang. Die meistverwendeten CMS sind momentan Wordpress, Drupal  und Joomla.

\section{ECMS}

Enterprise-Content-Management erweitert die Funktionalit�t eines CMS auf die Ebene einer kompletten Organisation. Das CMS und die resultierende Website sind also eine Komponente eines ECMS. Dabei k�nnen diese Systeme je nach Unternehmen sehr unterschiedliche Funktionen �bernehmen. Im Wesentlichen helfen sie dabei die Arbeit und Zusammenarbeit innerhalb einer Organisation zu vereinfachen. Konkrete Funktionen �hneln denen des CMS oft sehr, wie das Verwalten von Dateien, beschr�nken sich jedoch oft auf ein internes Netzwerk von Mitarbeitern.