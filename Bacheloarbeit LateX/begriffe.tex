\chapter{Begriffskl�rungen}

Zum vollen Verst�ndnis der Arbeit werden grundlegende Kenntnisse des Software Engineering und der Web Technologien vorausgesetzt. Essentielle Begrifflichkeiten und Technologien werden nun erkl�rt.

\section{CMS und WCMS}

Ein CMS (Abk.\textsc{C}ontent \textsc{M}anagement \textsc{S}ystem ), oder auf Deutsch Inhaltsverwaltungssystem, ist eine Software, die bei der Erstellung, Pflege und Planung von Content helfen kann. Das trifft vor Allem auf gemeinschaftlich genutzte Systeme zu. Heutzutage trifft man solche CMS zumeist im Web, woraus sich der Begriff WCMS (Abk. \textsc{W}eb \textsc{C}ontent \textsc{M}anagement \textsc{S}ystem) ergibt. Diese erm�glichen die Erstellung und Bearbeitung multimedialer Inhalte auf Webseiten ohne Programmierkenntnisse. So kann beispielsweise ein Redakteur, ohne viel M�he, News auf einer Webseite ver�ffentlichen. 
Content Management Systeme umfassen meist folgende Features.\cite{Nix2005}

\paragraph{Typische Features eines CMS}
\begin{itemize}
	\item Nutzerrollen, Rechte- und Gruppenmanagement
	\item Modularit�t und Erweiterbarkeit
	\item Erstellung von Workflows
	\item Erstellung von contentunabh�ngigen Templates
	\item Versionsverwaltung von Content
	\item Integrierte Suche von Content
	\item Ver�ffentlichung von Content
\end{itemize}


Fast immer gliedern sich Web Content Management Systeme in ein Backend und ein Frontend. Im Backend k�nnen sich nur bestimmte Nutzer, wie Administratoren und Autoren/Redakteure einloggen um hier die Seite und deren Inhalte zu verwalten. Daf�r ist kein extra Programm n�tig, das Backend l�sst sich meist bequem durch den Browser erreichen. Das Frontend ist die Webseite, welche �ffentlich zug�nglich ist und jene Inhalte f�r die Besucher der Webseite pr�sentiert.
Die Liste von bestehender CMS Software ist sehr lang. Die meistverwendeten WCMS sind momentan Wordpress, Drupal  und Joomla.\footnote{http://trends.builtwith.com/cms}

Obwohl WCMS der pr�zisere Begriff ist, werden diese, aufgrund ihrer Verbreitung, oft mit dem Oberbegriff CMS betitelt. Auch in dieser Arbeit, wenn nicht explizit erw�hnt, werden die Begriffe synonym verwendet. 

\section{ECMS}

Enterprise-Content-Management Systeme erweitern die Funktionalit�t eines CMS auf der Ebene einer kompletten Organisation. Das CMS und die resultierende Website sind oft eine Komponente eines ECMS. Dabei k�nnen diese Systeme je nach Unternehmen sehr unterschiedliche Funktionen �bernehmen. Im Wesentlichen helfen sie dabei die Arbeit und Zusammenarbeit innerhalb einer Organisation zu vereinfachen. Konkrete Funktionen �hneln denen des CMS oft sehr, wie das Verwalten von Dateien und bedienen dabei ein internes Netzwerk von Mitarbeitern.\cite{aiimECMS}

\section{Web Portal}

Ein Web Portal ist eine Webanwendung, die Informationen aus verschiedenen Quellen uniform b�ndelt. Einzelne Quellen, auch Portlets genannt, bekommen meist einen Bereich der Website-Oberfl�che zugewiesen. Ein typisches Merkmal ist die Anpassung der Portlets durch Drag und Drop. Sehr oft sieht man Web Portale im Intranet von Organisationen als zentrale Anlaufstelle mit personalisierten Inhalten, einheitlichem Design und einem einzigen Login f�r alle Systeme unterhalb eines Web Portals. Erst beim Transfer bestehender Anwendungen in ein Web Portal kommen deutliche Nachteile zum Vorschein, da dieser Transfer sehr komplex werden kann.

Sowohl WCMS als auch Web Portale erstellen Webseiten und �berlappen sich in ihrer Funktion. Web Portale sind dabei mehr auf die Applikationen fokussiert und WCMS mehr auf den Content.