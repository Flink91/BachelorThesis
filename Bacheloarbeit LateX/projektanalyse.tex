\chapter{Projektanalyse und Spezifikation}
\label{kap-projektanalyse}

\begin{quote}
	\emph{\textquotedblleft Those who fail to plan, plan to fail\textquotedblright} - Winston Churchill
\end{quote}
\medskip
\medskip
Die Planungsphase eines Projekts ist sehr wichtig und erspart, sofern richtig durchgef�hrt, eine Menge Arbeit. Diese Arbeitsersparnis w�chst exponentiell mit der Gr��e des Projekts und trifft somit auch besonders auf das gr��ere Projekt der WegE Plattform zu. Der folgende Abschnitt wird in Manier des klassischen Software Engineering das Projekt analysieren. Hier geht es vor Allem um das Erkennen und die Spezifikation der Anforderungen an die Plattform, denn diese beeinflussen nat�rlich direkt die technische Anpassung, die in Kapitel 6 vorgenommen wird.

\section{Der Prozess der Softwareentwicklung}

Schon in den 1960ern erkannte man die folgende Aufteilung bei der Entwicklung von Software.
Dieser generische Softwareentwicklungprozess, oder auch Systems Development Life cycle genannt, besteht aus f�nf Schritten:

\begin{enumerate}
	\item Specification (Analyse)
	\item Architecture \& design (Entwurf) 
	\item Implementation (Implementierung) 
	\item Testing (Test)
	\item Maintenance (Wartung)
\end{enumerate}

Jeder Bereich ist f�r sich ein eigenes Fachgebiet und kann je nach Gr��e und Art des Projekts vom Aufwand stark variieren. Dennoch finden sich immer diese f�nf Teile in einem zyklischen Ablauf wieder, weshalb es sich als eine gute Basis zur Orientierung innerhalb eines Projekts anbietet. In dieser Bachelorarbeit steht vor Allem die Analyse und der Entwurf im Vordergrund. Zwar werde Ich parallel zu dieser Arbeit eine exemplarische Webseite implementieren, aber nicht die eigentliche WegE Plattform umsetzen. So spielt sich diese Arbeit aus der Sicht des WegE Projekts vor Allem im zweiten Teil ab. Die Wartung spielt auch in dem Sinne eine gewisse Rolle, als dass der Aufwand daf�r bei der Implementation m�glichst klein gehalten werden soll. 

Da nun klar ist, dass vor Allem die Analyse und der Entwurf des WegE Projekts in dieser Arbeit eine Rolle spielen, k�nnen diese Teile nun vertieft werden.

\section{Analyse - Project Blastoff}

In der Analyse-Phase geht es vor Allem darum, zu verstehen was gebaut werden soll. Dabei hilft das sehr n�tzliche Konzept des Project Blastoff.

Im Project Blastoff wird die Realisierbarkeit �berpr�ft, diverse Infos gesammelt und letztendlich entschieden ob das Projekt begonnen werden soll. Konkrete Outputs des Project Blastoff sind: Risikoanalyse, Kostenanalyse, Terminologiesammlung, Stakeholdermap, Ziele, Einschr�nkungen, Umfang/ Eckpfeiler des Projekts.

F�r dieses Projekt werden nicht alle Teile ausgearbeitet, das Konzept des Project Blastoff dient mehr als Einordnung in den Entwicklungsprozess.

\section{Anforderungen an das WegE Projekt}

F�r die Erhebung der Anforderungen an ein System gibt es verschiedene M�glichkeiten. 
\section{Entwurf - Software Design}



