\chapter{Projektanalyse und Spezifikation}
\label{kap-projektanalyse}

\begin{quote}
	\emph{\textquotedblleft Those who fail to plan, plan to fail\textquotedblright} - Winston Churchill
\end{quote}
\medskip
\medskip
Die Planungsphase ist sehr wichtig und erspart, sofern richtig gemacht, eine Menge Arbeit. Das trifft besonders auf ein gr��eres Projekt, wie die WegE Plattform zu. Der folgende Abschnitt wird in Manier des klassischen Software Engineering das Projekt analysieren. Besonders wichtig sind dabei die Anforderungen an die Plattform, denn diese beeinflussen nat�rlich direkt die technische Anpassung, die in Kapitel 6 vorgenommen wird.

\section{Der Prozess der Softwareentwicklung}

Vorweg einige n�tzliche Konzepte aus Software Engineering.

Schon in den 1960ern entstand die folgende Aufteilung beim Entwickeln von Software.
Dieser generische Softwareentwicklungprozess, oder auch Software Development Life cycle genannt, besteht aus f�nf Schritten:

\begin{enumerate}
	\item Specification (Analyse)
	\item Architecture \& design (Entwurf) 
	\item Implementation (Implementierung) 
	\item Testing (Test)
	\item Maintenance (Wartung)
\end{enumerate}

Jeder Bereich ist f�r sich ein eigenes Fachgebiet und kann je nach Gr��e und Art des Projekts vom Aufwand variieren. Dennoch finden sich immer diese f�nf Teile in einem zyklischen Ablauf wieder, weshalb es sich als eine gute Basis zur Orientierung innerhalb eines Projekts anbietet. In dieser Bachelorarbeit steht vor Allem die Analyse und der Entwurf im Vordergrund. Zwar werde Ich parallel zu dieser Arbeit eine exemplarische Webseite implementieren, aber nicht die eigentliche WegE Plattform umsetzen. So spielt sich diese Arbeit aus der Sicht des WegE Projekt vor Allem im zweiten Teil ab. Die Wartung spielt auch in dem Sinne eine gewisse Rolle, als dass der Aufwand daf�r bei der Implementation m�glichst klein gehalten werden soll. 

Da nun klar ist, dass vor Allem die Analyse und der Entwurf des WegE Projekts in dieser Arbeit eine Rolle spielen, k�nnen diese nun vertieft werden.

\section{Analyse - Project Blastoff}

In der Analyse-Phase geht es vor Allem darum, zu verstehen was gebaut werden soll. Dabei hilft das sehr n�tzliche Konzept des Project Blastoff.




\section{Entwurf - Software Design}



