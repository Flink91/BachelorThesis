\chapter{Anhang -Typo3 Beispielinstallation}
\label{Anhang0}

In diesem Unterpunkt wird der tats�chliche Installationsprozess der Testumgebung, inklusive einem Blick auf den Code, dokumentiert. Diese erlaubt es am Anschaulichsten n�tige anf�ngliche Konfigurationen zu erkl�ren. Da sich diese Installationsanleitung sehr praktisch gestaltet, sollten die Versionen der genutzten Software beachtet werden, da hier schnell �nderungen auftreten k�nnen. Die Testumgebung l�uft mit Typo3 7.6 LTS und Wamp 3.0.0 mit PHP 5.6. Wichtig hierbei ist vor Allem die Typo3 Version. Es wird mit Bootstrap Version 3 gearbeitet.

\paragraph{Typo 3 Installationsschritte}
\begin{enumerate}
	
	\item Typo3 7.6 LTS herunterladen.
	\item Wamp oder vergleichbares Programm herunterladen und installieren.
	\item Den Typo3 Ordner in den \texttt{www}-Ordner von Wamp (htdocs bei XAMPP) verschieben.
	\item Nach dem Starten von Wamp zu \texttt{localhost/\textit{typo3Ordner}} navigieren.
	\item Es erscheint ein Installationsguide und wahrscheinlich eine Reihe von Fehlern, die zu beheben sind.
	\item Die \texttt{php.ini} �ffnen
	\subitem Die Variable \texttt{memory\_limit} erh�hen auf 64 oder h�her.
	\subitem Die Variable \texttt{upload\_max\_filesize} auf mindestens 10MB erh�hen.
	\subitem Die Variable \texttt{max\_execution\_time} auf 240 setzen.
	\item Die OpenSSL Extension muss als Systemvariable gesetzt werden. Unter Systemvariablen die Variable mit dem Namen \texttt{OPENSSL\_CONF} und dem Wert des Pfades zur openssl.cnf angeben. Danach eventuell Computer neu starten.
	\item Die PHP Extension Fileinfo geht oft nicht. In der php.ini das Kommentar-Semikolon vor\\ \texttt{extension=php\_fileinfo.dll} entfernen.
	\item Windows Apache Thread Stack Size Fehler. Dieser kann in der httpd.conf Datei angepasst werden. Hier folgenden Codeschnipsel ans Ende der Datei kopieren:\\ \texttt{<IfModule mpm\_winnt\_module> ThreadStackSize 8388608 </IfModule>}
	\item Alle Dienste von Wamp neu starten und zum Installationstool von Typo3 zur�ckkehren. Im n�chsten Schritt wird eine Datenbank ben�tigt. Will man diese selbst anlegen hilft das Tool phpmyadmin unter localhost/phpmyadmin. Hier l�sst sich mit einem Klick eine leere Datenbank anlegen.
	\item Typo3 verlangt beim Anlegen der Datenbank auch den Port. Dieser l�sst sich in der Konsole mit dem Befehl netstat -a -o auslesen. Hierzu vergleicht man die angezeigte PID mit dem Wamp Prozess im Taskmanager. F�r gew�hnlich ist der Port eine Zahl um die 3000.
	\item Nun l�sst sich das Backend, durch das Anh�ngen von /typo3 in der Adresszeile, aufrufen.
	
\end{enumerate}
