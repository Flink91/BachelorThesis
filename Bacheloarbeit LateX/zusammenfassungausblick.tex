\chapter{Zusammenfassung} 
\label{kap-zusammenfassung}%

In dieser Arbeit wurde behandelt, wie ein web-basiertes Projekt richtig angegangen werden sollte. Es wurde gezeigt, wie sich der Prozess der Softwareentwicklung, trotz sehr unterschiedlicher Modelle immer in f�nf Phasen einteilen l�sst. Analyse, Entwurf, Implementierung, Testen und Wartung. Alle Schritte wurden, am Beispiel der WegE Plattform der Universit�t Bamberg, welches aktuell umgesetzt wird, konkretisiert. Zu Beginn eines Projekts spielt das Requirements Engineering eine entscheidende Rolle, bei welchem alle n�tigen Anforderungen eines Projekts sammelt. Sind bei einigen Requirements Unklarheiten �ber Machbarkeit dieser, ist es sinnvoll diese prototypisch zu testen und die Ergebnisse dessen zu validieren. Eine solche Liste mit unklaren Requirements wurde in dieser Arbeit getestet. Vor dem Test wurde erkl�rt, wie passende Technologien, genauer CMS, f�r ein Projekt bestimmt werden. Neben technischen Daten spielen hier auch Budget und Vorwissen eine gro�e Rolle. Diese Kriterien wurden an drei ausgew�hlten CMS untersucht. F�r das WegE Projekt fiel die Wahl auf Typo3, welches ausf�hrlich vorgestellt wird um dann die zuvor ausgew�hlten Requirements zu implementieren und zu testen. Aus Gr�nden der Effizienz wurden dabe vor Allem die M�glichkeiten von Extensions untersucht. Nach der Implementierung lie� sich die Validierung dieser Anforderungen durchf�hren und die erhaltenen Ergebnisse k�nnen in die weitere Planung des WegE Projekt flie�en.

\section{Erkenntnisse}

Typo3 ist ein flexibles und sehr m�chtiges Web Content Management System. Als jemand mit fundierten Kenntnissen in der Webentwicklung und dem WCMS Worpress, war Ich dennoch von der Gr��e erschlagen. Typo3 ist nicht f�r Projekte geeignet, die alleine bew�ltigt werden k�nnen. F�r gro�e Projekte ist es jedoch eine fantastische Wahl und je mehr ich in Typo3 arbeitete, desto klarer wurden die M�glichkeiten und Vorz�ge. Durch die steile Lernkurve von Typo3 sind diese f�r Anf�nger n�mlich nicht unbedingt ersichtlich. 
Einer der leicht ersichtlichen Vorteile ist das Extension Repository. Sich durch wenige Klicks neue Funktionalit�t ins System zu laden ist �berzeugend. Die Auswahl und Qualit�t der Extensions ist �berwiegend gut. Allerdings sind viele der Extensions keine komplette Out-of-the-box-L�sung und ben�tigen eine Portion Eigeninitiative. Vor Allem wenn es an spezielle Anpassungen geht. Dies ist f�r schnelle, kleine Projekte von Nachteil, f�r gro�e Projekte gewinnt man dadurch mehr Kontrolle �ber die Extensions. 
Beim Kapitel zur Projektanalyse wurde sehr deutlich welche Vorteile gut Planung haben kann. Doch auch einige negative Aspekte fielen auf. Bei sehr vielen Konzepten des Software Engineering handelt es sich um Richtlinien, welche nur von Vorteil sind, wenn sie an der richtigen Stelle angewendet werden. So fiel mir bei der Recherche dazu auf, dass auch zu viel geplant werden kann und die Planung damit dem Projekt sogar schaden kann. Ist der Ausgang des Projekts jedoch kritisch und die Projektdomain neu, sollte definitiv nicht auf gutes Software Engineering verzichtet werden. 


\section{Ausblick auf die Zukunft von CMS Entwicklung und der WegE Plattform}

Das Erstellen von Web Plattformen ist so leicht wie noch nie.